%-----Headers and footers-----
\usepackage{fancyhdr}
%\setlength{\headheight}{16pt}
%\fancyhf{}%Clear default settings
%\renewcommand{\headrulewidth}{1pt}
%\lhead{\rightmark}
%\cfoot{\thepage}

%-----Table of contents-----
\renewcommand{\contentsname}{Table of Contents}
\setcounter{tocdepth}{3}

%-----Line spacing-----
\usepackage{setspace}
\onehalfspacing
%\doublespacing
%\raggedbottom

%-----Margins-----
\usepackage[top=1in,bottom=1in,left=1in,right=1in,headheight=15pt]{geometry}

%-----Links----
\usepackage[colorlinks=true, linkcolor=black]{hyperref}

%-----Ability to set text color-----
%\colorbox{yellow}{foo foo foo}
%\usepackage{color}

\usepackage{amsmath}

%------Table row colors------
%Place this command just before a tabular environment
%\rowcolors{1}{green}{pink}
%Use these in a tabular environment
%\hiderowcolors
%\showrowcolors
\usepackage[table]{xcolor}

%-----Let's the user added arbitrary notes for future document work----
%https://www.ctan.org/pkg/todonotes?lang=en
%\todo[]{todo text}
%\missingfigure{note text instead of figure}
%Note: If multiple TODOs are on a single page it's likely to generate a "marginpar on page X moved."
%This lets the author know that margins have been adjusted to handle the TODOs.
%Pass 'disable' flag to turn off all todo notes in the document
%\usepackage[disable]{todonotes}
%\usepackage{todonotes}

%-----Code syntax highlighting-----
%minted provides a wrapper around Pygments for syntax highlighting
%http://pygments.org/
%https://github.com/gpoore/minted/blob/master/source/minted.pdf
\usepackage{minted}

%-----Pseudocode/Algorithsm-----
%http://tug.ctan.org/macros/latex/contrib/algorithmicx/algorithmicx.pdf
\usepackage{algorithm}
%\usepackage{algorithmic}
\usepackage{algpseudocode}


%----Graphics-----
\usepackage{graphicx}
%Add additional paths by closing them in braces {/path/to/somewhere/}
%Ending / is required
\graphicspath{ {./diagrams/} }

%----SVG Graphics-----
%\usepackage{svg}
%	\includesvg{path/to/file/svg}

%---Control of the number of columns on the page
\usepackage{multicol}

%----Abbreviations and symbols listing---
\usepackage{enumitem}
\newlist{abbrv}{itemize}{1}
\setlist[abbrv,1]{label=,labelwidth=1in,align=parleft,itemsep=0\baselineskip,leftmargin=!}

%-----Citations-----
%http://ctan.sharelatex.com/tex-archive/macros/latex/contrib/biblatex/doc/biblatex.pdf
\usepackage[american]{babel}
\usepackage{csquotes}
%\usepackage[style=apa,backend=biber]{biblatex}
\usepackage[style=apa,citestyle=apa,backend=biber]{biblatex}
\DeclareLanguageMapping{american}{american-apa}
%Had been using this verson
%\usepackage[backend=biber,style=authoryear,citestyle=authoryear,maxcitenames=1,natbib]{biblatex}

%\usepackage[backend=biber,style=apa,citestyle=authoryear,maxcitenames=1,natbib]{biblatex}
%\addbibresource{text/references.bib}


%----Remove excess whitespace around chapter titles----
%\usepackage{titlesec}
%\titleformat{\chapter}[display]   
%{\normalfont\huge\bfseries}{\chaptertitlename\ \thechapter}{20pt}{\Huge}   
%\titlespacing*{\chapter}{0pt}{-50pt}{40pt}

\usepackage{etoolbox}
\makeatletter
\patchcmd{\@makechapterhead}{50\p@}{0pt}{}{}
\patchcmd{\@makeschapterhead}{50\p@}{0pt}{}{}
\makeatother


%-----Change to arial font for default as required by school formatting----
%\renewcommand*\rmdefault{arial}

\usepackage[titletoc]{appendix}

