%\chapter{UAV Model}
\section{UAV Model}
\label{sec:uav_model}
Each UAV within the simulation is a self contained entity capable of acting and thinking on its own accord.  No UAV requires the presence of any other UAV or the aid of any external actor.  The swarm is made of multiple independent UAVs that share information and in doing so coordinate actions to complete the mission.  The simulation supports an arbitrary number of user defined UAV types (or platforms) with unique payloads.  Payload configurations for this work are shown in appendix~\ref{sec:pyldConfigs}.


%\section{UAV Kinematics}
%\subsection{UAV Kinematics}
Each UAV type has a fixed speed and turning radius.  The fixed speed represents the aircraft's standard cruising speed that it will maintain throughout the simulation.  An example set of UAV type kinematic configuration data is shown in Table~\ref{tab:uavKinematic}.

\begin{table}[H]
	\caption{UAV kinematic definitions}
	\centering
	\rowcolors{1}{lightgray}{white}
	\label{tab:uavKinematic}
	\begin{tabular}{|p{1cm}|p{2cm}|p{1cm}|}
		\hline
		UAV Type & Turning Radius (m) & Speed ($\frac{m}{s}$)\\ \hline
		0 & 150 & 100 \\ \hline
		1 & 300 & 60 \\ \hline
	\end{tabular}
\end{table}

UAVs know their current 2D coordinate and heading.  This information is used to compute a path to any destination location and orientation.  In formal mathematics terms the kinematic model of a UAV is defined as \dots

\begin{align}
\dot{x} &= v \cos(\psi) \label{eq:uavChngX}\\
\dot{y} &= v \sin(\psi) \label{eq:uavChngY}\\
\dot{\psi} &= \{-constant, 0, constant\} \label{eq:uavTurnRate}\\
\dot{v} &= 0 \label{eq:uavAccel}\\
\psi_{constant} &= \frac{v*\sin(\pi)}{r} \label{eq:uavTurnRateDeriv}
\end{align}

\dots where $x$ and $y$ represent the UAV's current position on a Cartesian plane, $v$ is the UAV's fixed speed, $\psi$ is the UAV's heading, and $r$ is the UAV's turning radius.  Equations~\ref{eq:uavChngX} and \ref{eq:uavChngY} enforce that a UAV's position changes relative to its speed on a continuous trajectory.  Equation~\ref{eq:uavTurnRate} forces the UAVs to always turn at a known rate or not at all.  This allows for simplified trajectory planning using the equations for Dubin's Curves described in \citet{dubins}.  Equation~\ref{eq:uavAccel} states that the UAVs move at a constant speed.  The constant for the turn rate is computed as shown in equation~\ref{eq:uavTurnRateDeriv}.  The world's terrain is modeled as a flat bald Earth.  This set of kinematic equations is a common model for simulating aircraft movement on a 2D plane and can be seen elsewhere in \citet{beard}, \citet{finke}, \citet{mclainBeard}, and \citet{coopPathBook}.

