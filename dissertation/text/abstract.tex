\begin{abstract}
The goal of this research was to create a decentralized algorithm for controlling multiple heterogeneous Unmanned Air Vehicles (UAVs) in ``search and destroy'' missions that minimized communication interactions.  This allows the swarm to cooperatively operate in denied, intermittent, and high latency communication environments such as battlefield conditions or in first responder situations.  The new Implicit Territorial Swarming algorithm only requires one-way communication between its members.  This is an unusual choice since one of the primary benefits of swarms is their ability to actively cooperate, share information, and negotiate decisions which is usually performed by N-way conversations.  %Restricting communications to one-way only greatly simplifies the process of retrofitting preexisting UAVs with the capability to form a cooperative swarm and avoids many common problems found in mobile ad hoc networks.
	
	
%The goal of this research was to develop a decentralized algorithm for controlling multiple heterogeneous Unmanned Air Vehicles (UAVs) in ``search and act'' scenarios that were not originally designed to cooperate as a team.  Existing UAVs already have logic that controls their flight behaviors, how they accomplish goals, how they detect tasks to perform, and how they encode their data.  Attempting to modify this existing logic to cooperate with other unknown and different aircraft types is a burdensome undertaking.%  UAVs used in safety critical applications undergo severe scrutiny during development before they are allowed to perform their missions.  Therefore any changes to these types of vehicles is even more difficult yet these are the types of vehicles that would benefit the most from cooperative swarm capabilities.

%To minimize the changes necessary to retrofit preexisting legacy type UAVs with swarming capabilities this work developed a decentralized swarm control algorithm that only requires one-way communication between its members.  This is an unusual choice since one of the primary benefits of swarms is their ability to actively cooperate, share information, and negotiate decisions which is usually performed by N-way conversations.  Restricting communications to one-way only greatly simplifies the process of retrofitting preexisting UAVs with the capability to form a cooperative swarm and avoids many common problems found in mobile ad hoc networks.  Ideally very few modifications are required to preexisting UAV control logic and the new swarming logic can be an augmentation to the preexisting command inputs and logic outputs of the UAV.  The work in this dissertation seeks to maintain the benefits of swarming while reducing the required infrastructure.

%A new decentralized swarm control algorithm that mimics the behavior of territorial animals is created and simulated with a virtual fleet of reconnaissance and attack UAVs.  The fleet is instructed to complete a search-and-destroy mission in an unknown environment with an unknown number of static and mobile targets.  The experiment varies the communication range of the swarm members and shows that even with severely limited communication ranges one-way communications still allow for successful mission completion with swarming.

A simulated fleet of heterogeneous UAVs is created that is composed of reconnaissance and attack vehicles.  Using the new control algorithm the fleet is instructed to complete a search-and-destroy mission in an unknown environment with an unknown number of static and mobile targets.  The algorithm is stressed by running multiple simulations with varying communication ranges.  The communication range of the vehicles restricts how well the swarm can communicate and therefore how well they can coordinate.  The simulations show that even with severely limited communication ranges and low density of UAVs that the new swarming algorithm can still complete the missions in reasonable amounts of time compared to a fleet with a global communication range.

\end{abstract}

Keywords: swarming, distributed, artificial intelligence, control, communication

\newpage


