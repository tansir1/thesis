\chapter{Introduction}
A common task for teams of agents is to ``search and act.''  The team members must search an area for a target object and then perform an action once the objective is found.  This commonly referred to as ``search and rescue,'' ``search and destroy,'' or ``find and report.''  These types of missions are commonly carried out by military forces, first responders, and commercial entities such as in precision agriculture.  These organizations have unmanned air vehicles (UAVs) that are typically self contained systems between the vehicle and ground control station.  Team coordination amongst the members of the fleet is accomplished by out-of-band communications requiring operators to manually copy data between systems or by agreeing to a common messaging protocol requiring all UAV systems to be developed and deployed in lock-step fashion.  

Coordinating the efforts of a team of aircraft in a small airspace is difficult and dangerous.  Add in limited or disrupted communications such as in the case of military and first responder applications and the amount of risk involved dramatically increases.  This mission management work requires many people to operate together and all the supporting equipment that goes with it. Swarming control algorithms can help automate this coordination process.  Automation will allow much faster and much more accurate modeling of the region compared to a group of people manually moving data.  Swarming logic can serve as air traffic controllers, pilots, and mission controllers simultaneously.  This means aircraft can fly closer together and strategically allocate themselves to tasks without interfering with one another.  More aircraft in the mission airspace and more strategic tasking allocation means missions can be completed faster.

Unfortunately the unmanned vehicle industry is still maturing.  Most unmanned systems are closed environments that do not integrate with others.  Even sharing data amongst the same type of systems is uncommon.  Newer unmanned systems are trying to integrate swarming capabilities as a core component of the system but these activities are largely in the research phase still.  At the moment there are many already deployed UAVs that perform duties that could benefit from cooperative swarming behavior.  Therefore there is a need for swarming control logic that is lightweight and non-invasive that can be augmented onto existing UAV systems.

To minimize the changes necessary to retrofit preexisting legacy type UAVs with swarming capabilities this work proposes a new type of decentralized swarm control algorithm with an emphasis on minimizing control logic and communication requirements.  The algorithm does not require specialized swarm members with extra capabilities to aid swarm management, it does not require multi-step negotiations between individuals, and it does not require members of the swarm to maintain communication network paths between platforms.  The lack of these common swarming requirements means the aircraft only need one-way communication between each other and allows for the use of cheap, lightweight, and low power CPUs.  Ideally the new swarming logic can be augmented to the preexisting command input system on the UAV minimizing any changes to the internal UAV logic and hardware.  This helps to prevent and eliminate risky rework of the flight control logic and helps to reduce issues triggering flight safety qualification reviews.

This paper proposes a new minimalistic decentralized swarm control algorithm by mimicking the behavior of territorial animals.  A simulated fleet of heterogeneous UAVs is created composed of reconnaissance and attack vehicles.  The fleet is instructed to complete a search-and-destroy mission in an unknown environment with an unknown number of static and mobile targets.  The algorithm is stressed by running multiple simulations with varying communication ranges.  The communication range of the vehicles restricts how well the swarm can communicate and therefore how well they can coordinate.  The simulations show that even with severely limited communication ranges and low density of UAVs that the new swarming algorithm can still complete the missions in reasonable amounts of time compared to a fleet with a global communication range.

Chapter~\ref{chap:relWork} discusses previous work in cooperative swarms that influenced this experiment.  Chapter~\ref{chap:worldScenModel} explains the model used to simulate the virtual world and fleet.  Chapter~\ref{chap:myModel} constructs the new territorial swarming model and justifies the reasoning behind the idea.  Chapter~\ref{chap:results} analyzes the results of the algorithm as simulated through the virtual worlds.  Chapter~\ref{chap:conclusion} provides some conclusions about the work and ideas for future investigations.

