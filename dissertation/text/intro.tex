\chapter{Introduction}
As described in~\textcite{dod_uas_roadmap}, the unmanned vehicle industry is still maturing.  Most unmanned systems are closed environments that do not integrate with others.  Even sharing data amongst the same type of systems is uncommon.  Despite the current status of the industry cooperative UAV missions is a goal of the United States military.

Situations involving military force and first responders do not always have the luxury of a nearby accessible and reliable communication infrastructure system.  In order for UAVs to actively cooperate during a mission they will have to deal with denied, intermittent, and high latency communication environments.  Maintaining reliable communications in battlefield conditions is an active problem with many possible solutions \parencites{tacom_modeling, gapr2_dtn_routing, unmet_wireless_challenges}. Most technical approaches require additional software overhead \parencite{p2p_tac_edge, reliable_tac_com} or pre-determined anti-jamming procedures \parencite{anti_jam_technique}.  No matter the mechanism these existing approaches attempt to circumvent jamming or re-route around a problem area.

Another technical approach is explored in this dissertation.  Instead of trying to improve the communication network can we reduce communication requirements and still cooperate? The aim of this dissertation research is to create a new type of decentralized swarm control algorithm that minimizes control logic and communication requirements by mimicking the behavior of territorial animals such that control is implied instead of explicitly coordinated.  The algorithm does not require specialized swarm members with extra capabilities to aid swarm management, it does not require multi-step negotiations between individuals, and it does not require members of the swarm to maintain communication network paths between platforms.  The lack of these common swarming requirements means the aircraft only need one-way communication between each other which is more likely to succeed in denied, intermittent, and high latency communication environments.

Chapter~\ref{chap:relWork} reviews previous work in cooperative swarms that provided the foundation for the research conducted in the present dissertation.  Chapter~\ref{chap:worldScenModel} explains the virtual world model used in the simulation, constructs the new Implicit Territorial Swarming algorithm, and justifies the reasoning behind the idea.  Chapter~\ref{chap:results} presents empirical results and Chapter~\ref{chap:discussion} discusses the results of the algorithm as simulated through the virtual worlds. It also examines known limitations and ideas for future investigation.  Chapter~\ref{chap:conclusion} provides conclusions about the work.

%A common task for teams of UAVs is to ``search and act.''  The team members must search an area for a target object and then perform an action once the objective is found.  This commonly referred to as ``search and rescue,'' ``search and destroy,'' or ``find and report.''  These types of missions are commonly carried out by military forces, first responders, and commercial entities such as in precision agriculture.  These organizations use UAVs that are typically self contained systems between the vehicle and ground control station.  Team coordination amongst the members of the fleet is accomplished by out-of-band communications requiring operators to manually copy data between systems or by agreeing to a common messaging protocol requiring all UAV systems to be developed and deployed in lock-step fashion.  

%Coordinating the efforts of a team of aircraft in a small airspace is difficult and dangerous.  Add in limited or disrupted communications such as in the case of military and first responder applications and the amount of risk involved dramatically increases.  This mission management work requires many people to operate together and all the supporting equipment that goes with it. Swarming control algorithms can help automate this coordination process.  Automation will allow much faster and much more accurate modeling of the region compared to a group of people manually moving data.  Swarming logic can serve as air traffic controllers, pilots, and mission controllers simultaneously.  This means aircraft can fly closer together and strategically allocate themselves to tasks without interfering with one another.  More aircraft in the mission airspace and more strategic tasking allocation means missions can be completed faster.

%Unfortunately, as described in~\textcite{dod_uas_roadmap}, the unmanned vehicle industry is still maturing.  Most unmanned systems are closed environments that do not integrate with others.  Even sharing data amongst the same type of systems is uncommon.  Newer unmanned systems are trying to integrate swarming capabilities as a core component of the system but these activities are largely in the research phase still.  At the moment there are many already deployed UAVs that perform duties that could benefit from cooperative swarming behavior.  Therefore there is a need for swarming control logic that is lightweight and non-invasive that can be augmented onto existing UAV systems.

%The aim of this dissertation research is to create a new type of decentralized swarm control algorithm that minimizes the changes necessary to retrofit preexisting legacy type UAVs with swarming capabilities.  The new algorithm emphasizes minimizing control logic and communication requirements by mimicking the behavior of territorial animals such that control is implied instead of explicitly coordinated.  The algorithm does not require specialized swarm members with extra capabilities to aid swarm management, it does not require multi-step negotiations between individuals, and it does not require members of the swarm to maintain communication network paths between platforms.  The lack of these common swarming requirements means the aircraft only need one-way communication between each other and allows for the use of cheap, lightweight, and low power CPUs.  Ideally the new swarming logic can be augmented to the preexisting command input system on the UAV minimizing any changes to the internal UAV logic and hardware.  This helps to prevent and eliminate risky rework of the flight control logic and helps to reduce issues triggering flight safety qualification reviews.

%Chapter~\ref{chap:relWork} reviews previous work in cooperative swarms that provided the foundation for the research conducted in the present dissertation.  Chapter~\ref{chap:worldScenModel} explains the virtual world model used in the simulation, constructs the new Implicit Territorial Swarming algorithm, and justifies the reasoning behind the idea.  Chapter~\ref{chap:results} presents and discusses the results of the algorithm as simulated through the virtual worlds. It also discusses known limitations and ideas for future investigation.  Chapter~\ref{chap:conclusion} provides conclusions about the work.

