\chapter{Introduction}
A common task for teams of agents is to ``search and act.''  The team members must search an area for a target object and then perform an action once the objective is found.  This commonly referred to as ``search and rescue,'' ``search and destroy,'' or ``find and report.''  These types of missions are commonly carried out by military forces, first responders, and commercial entities such as in precision agriculture.

In this work our team of agents is a swarm of unmanned air vehicles (UAVs) performing a search and destroy mission.  Traditionally, controlling a swarm of UAVs requires complex top-down global coordination or bottom-up local cooperation between agents actively negotiating who performs a task.  The control logic developed here is a decentralized bottom up algorithm designed for minimal communication.  Only undirected communication broadcasts are required by the UAVs.  There is no bidirectional communication between the agents.  

%The algorithm allows mission planners to determine the minimum communication range needed to complete the mission successfully through Monte Carlo simulations.  

%Controlling military UAVs typically requires multiple people per vehicle.  There’s usually a pilot and a payload operator that work together to manipulate the aircraft.  There is little to no direct data sharing between multiple UAVs and their crews which creates significant inefficiencies when trying to cooperate and perform missions together.  These inefficiencies are compounded in unstructured missions that require multiple aircraft to search an area together for unknown targets.  The research presented here created a distributed control algorithm that coordinates the actions of heterogeneous UAV platforms with limited communication abilities so they can cooperate while searching and engaging targets.
