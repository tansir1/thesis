%\chapter{Sensor and Weapon Models}
\section{Sensor and Weapon Models}

UAVs carry sensors and weapons.  Sensors are used for scanning the world to determine if a target is present.  If a target is found sensors are then used for tracking the target.  Weapons are used for destroying targets.  There are multiple types of sensors and multiple types of weapons in the simulation.  Each has unique characteristics.  Each type of UAV carries a specific payload package of sensors and weapons.  This allows the UAVs to specialize as surveillance platforms, strike platforms, or multi-role platforms.
% \textbf{TODO: Should detect vs confirm be described here or in UAV models?}

%\section{Payload Type Definitions}
\subsection{Payload Type Definitions}

Each type of sensor has a minimum and maximum detection range.  Anything that is too close would oversaturate the sensor and yield invalid results.  Anything to far away does not produce enough of a signal to be detectable.  Sensors can be mounted on a gimbal turret or in a fixed orientation on their host UAV.  If they are gimbaled the sensors cannot slew faster than a maximum slew rate that is unique to the sensor type. A sample set of data characterizing sensors can be seen in table~\ref{tab:sensorType}.

\begin{table}[H]
	\caption{Sensor type definitions}
	\centering
	\rowcolors{1}{lightgray}{white}
	\label{tab:sensorType}
	\begin{tabular}{|p{1cm}|p{1.5cm}|p{1cm}|p{1cm}|p{1.5cm}|}
		\hline
		Sensor Type & Field of View ($^{\circ}$) & Min Range (m) & Max Range (m) & Slew rate ($\frac{^{\circ}}{s}$)\\ \hline
		0 & 60 & 10 & 1100 & 30 \\
		1 & 45 & 50 & 700  & 10 \\
		\hline
	\end{tabular}
\end{table}

%Sensors have two scanning modes.  The normal mode is a wide angle scan set to a fixed field-of-view unique to the sensor type as shown in figure~\ref{fig:wide_angle_scan}.  The second mode is a focused scan of a small area.  This mode is meant to emulate a sensor that is zoomed in on something of interest.  This mode is used to confirm target identities and perform battle damage assessment in the simulation as shown in figure~\ref{fig:narrow_scan}.

Sensors have two scanning modes.  The normal mode is a wide angle scan set to a fixed field-of-view unique to the sensor type.  The second mode is a focused scan of a small area.  This mode is meant to emulate a sensor that is zoomed in on something of interest.  This mode is used to confirm target identities and perform battle damage assessment in the simulation.  Figure~\ref{fig:scan_sizes} shows a UAV in the bottom right performing a wide area scan and another UAV in the center of the image performing a focused scan on a recently destroyed target.  Note the size difference between the sensor footprints.


\begin{figure}[H]
	\centering
	%	\includegraphics[width=\linewidth,height=4in,keepaspectratio=false]{averages.png}
	%	\includegraphics[scale=0.27]{wide_angle_scan.png}
	\includegraphics[width=\linewidth]{scan_sizes.png}
	\caption{Comparison of sensor footprints}
	\label{fig:scan_sizes}
\end{figure}

%\begin{figure}[H]
%	\centering
	%	\includegraphics[width=\linewidth,height=4in,keepaspectratio=false]{averages.png}
%	\includegraphics[scale=0.27]{wide_angle_scan.png}
%	\includegraphics[width=\linewidth]{wide_angle_scan.png}
%	\caption{A sensor performing a normal wide angle area scan.}
%	\label{fig:wide_angle_scan}
%\end{figure}

%\begin{figure}[H]
%	\centering
	%	\includegraphics[width=\linewidth,height=4in,keepaspectratio=false]{averages.png}
	%	\includegraphics[scale=0.27]{narrow_scan.png}
%	\includegraphics[width=\linewidth]{narrow_scan.png}
%	\caption{A sensor performing a focused narrow scan of a destroyed target}
%	\label{fig:narrow_scan}
%\end{figure}

Similarly, each weapon type has a minimum and maximum launch range.  If a target is too close the weapon will not have enough time to properly acquire the target and fuse itself.  If a target is too far away the weapon's propulsion system will fall short of a lethal distance.  Weapons have limited steering capabilities so the host UAV must align its heading to the target within an acceptable launch angle region. A sample set of data characterizing weapons can be seen in table~\ref{tab:weaponType}.  The launch angle region is defined as the area between two bearings referenced to the left and right of the heading of the UAV.  For example in table~\ref{tab:weaponType} weapon type 0 has a Launch Angle of $12^{\circ}$.  This means the bearing referenced from the nose of the UAV (boresight) to the target must be between $-12^{\circ}$ and $12^{\circ}$ before the UAV can launch the weapon.

% \todo{Do we care about a UAV being in the blast radius?}

\begin{table}[H]
	\caption{Weapon type definitions}
	\centering
	\rowcolors{1}{lightgray}{white}
	\label{tab:weaponType}
	\begin{tabular}{|p{1.4cm}|p{1.6cm}|p{1.2cm}|p{1.2cm}|}
		\hline
		Weapon Type & Launch Angle ($^{\circ}$) & Min Range (m) & Max Range (m)\\ \hline
		0 & 12 & 10 & 200 \\
		1 & 26 & 25 & 500 \\
		\hline
	\end{tabular}
\end{table}

%\section{Payload Probabilities}
\subsection{Payload Probabilities}
\label{sec:payload_probs}
This simulation uses realistic sensors in that they cannot perfectly detect targets or clear areas.  Each type of sensor has a probability of detecting if a target exists, a probability of confirming a target type, and an affinity for correctly estimating the heading of a target.  All of these attributes are user configurable and stored in a lookup table.  The values used for the experiment described in this paper are shown in table~\ref{tab:snsrTgtProb}.  These attributes are defined assuming perfect alignment with the target's \textit{Best Angle}.  Any variance from the target's \textit{Best Angle} will cause these probabilities and heading estimate to degrade.  For example, if a visible light spectrum camera is viewing a person's front we can see their face.  From this angle we can detect that we are looking at a person and confirm their identity.  If we orbit around behind the person we can still detect that a person is in front of the camera but the probability of confirming their identity correctly falls.

\begin{table}[H]
	\caption{Sensor to Target Probabilities}
	\centering
	\rowcolors{1}{lightgray}{white}
	\label{tab:snsrTgtProb}
	\begin{tabular}{|p{1cm}|p{1cm}|p{1cm}|p{1cm}|p{1cm}|}
		\hline
		Sensor Type & Target Type & Prob(Detect) & Prob(Confirm) & Heading Estimation Coefficient\\ \hline
		0 & 0 & 0.6 & 0.5 & 0.5 \\
		0 & 1 & 0.6 & 0.5 & 0.5 \\
		1 & 0 & 0.4 & 0.4 & 0.5 \\
		1 & 1 & 0.4 & 0.4 & 0.5 \\
		\hline
	\end{tabular}
\end{table}

Figure~\ref{fig:targetAngles} shows the gradient of performance scaling around a target based on the offset from the \textit{Best Angle}.  In this figure the target has a \textit{Best Angle} of $45^{\circ}$.  Any sensing or attacks at this angle will function with $100\%$ normal performance as defined in the probability lookup tables.  The direct opposite location at $225^{\circ}$ has a $0\%$ scaling value.  Any sensing or attack activities at this angle will not function at all.  Angles in between the \textit{Best Angle} and the opposite side will have their probabilities of success scaled by some percentage.

%\end{multicols}

\begin{figure}[H]
	\centering
	%	\includegraphics[width=\linewidth,height=4in,keepaspectratio=false]{averages.png}
	\includegraphics[scale=0.25]{targetAngles.png}
	\caption{Success scaling for a target with a best angle of $45^{\circ}$}
	\label{fig:targetAngles}
\end{figure}

Figure~\ref{fig:blue_red_angles} shows an example of two UAVs in blue interacting with a red target.  The relative angle between the target's heading and Blue 1's is close to the \textit{Best Angle} of $45^{\circ}$.  Blue 1 can sense or attack the target with little to no performance degradation.  The relative angle between the heading of the target and Blue 2 is much greater.  Therefore, Blue 2 will suffer a degradation in sensing and attack performance.

\begin{figure}[H]
	\centering
	%	\includegraphics[width=\linewidth,height=4in,keepaspectratio=false]{averages.png}
	\includegraphics[scale=0.25]{blue_red_angles.png}
	\caption{Success scaling per relative angles}
	\label{fig:blue_red_angles}
\end{figure}

%\begin{multicols*}{2}

Weapons have a similar lookup table functionality in regards to the probability of successfully destroying a target assuming it is struck from its \textit{Best Angle}.  Any deviation from the \textit{Best Angle} causes a drop in the probability of destruction.  For example, let's imagine a car.  They have crumple zones to absorb impact energy to protect the occupants.  In a head-on collision the occupants are likely to survive because the front of the car will crumple and absorb a significant amount of energy.  If a car is struck along its sides in the door panels there is less protection for the occupants.  In this case the \textit{Best Angle} for a car is along the sides in the door panels. A sample set of data showing the probability of destruction mapping for each weapon and target combination in this experiment is shown in table~\ref{tab:wpnTgtProb}.

\begin{table}[H]
	\caption{Weapon to Target Probabilities}
	\centering
	\rowcolors{1}{lightgray}{white}
	\label{tab:wpnTgtProb}
	\begin{tabular}{|p{1.5cm}|p{1.5cm}|p{3cm}|}
		\hline
		Weapon Type & Target Type & Prob(Destruction)\\ \hline
		0 & 0 & 0.75 \\
		0 & 1 & 0.75 \\
		1 & 0 & 0.75 \\
		1 & 1 & 0.75 \\
		\hline
	\end{tabular}
\end{table}

%\textbf{TODO: chance of misclassification. Chance of empty cell detection}


