\chapter{Sensor and Weapon Models}
UAVs carry sensors and weapons.  Sensors are used for scanning the world to determine if a target is present.  If a target is found sensors are then used for tracking the target.  Weapons are used for destroying targets.  There are multiple types of sensors and multiple types of weapons in the simulation.  Each has unique characteristics.  Each type of UAV carries a specific payload package of sensors and weapons.  This allows the UAVs to specialize as surveillance or strike platforms or to be a multi-role platform.

Each type of sensor has a minimum and maximum detection range.  Anything that is to close would oversaturate the sensor and yield invalid results.  Anything to far away does not produce enough of a signal to be detectable.  Sensors can be mounted on a gimbal turret or in a fixed orientation on their host UAV.  If they are gimbaled the sensors cannot slew faster than a maximum slew rate that is unique to the sensor type.  Sensors have two scanning modes.  The normal mode is a wide angle scan set to a fixed field-of-view unique to the sensor type.  The second mode is a focused scan of a small area.  This mode is meant to emulate a sensor that is zoomed in on something of interest.  This mode is used to confirm target identities and perform battle damage assessment in the simulation.

Similarly each weapon type has a minimum and maximum launch range.  If a target is too close the weapon will not have enough time to properly acquire the target and fuse itself. \todo{Do we care about a UAV being in the blast radius?}  If a target is to far away the weapon's propulsion system will fall short of a lethal distance.  Weapons have limited steering capabilities so the host UAV must align its heading to the target within an acceptable launch angle region.

Probabilities of everything

