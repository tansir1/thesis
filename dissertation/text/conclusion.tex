\chapter{Conclusions}
\label{chap:conclusion}

\textbf{TODO: REVISE HERE DOWN}

In this paper a model was presented for a heterogeneous swarm of UAVs to cooperate using only one-way communication by broadcasting their internal belief of the world and acting like territorial animals.  The UAVs were only allowed to broadcast singular messages with no back-and-forth exchanges or direct UAV to UAV exchanges.  The model was then stressed to see how it performs in a seek-and-destroy style mission with varying communication ranges.  It was found that even with very limited communication ranges and low densities of UAVs that the swarm was able to complete the mission in reasonable times compared to the baseline case of unlimited global communication range.

The real world application of the model can help to retrofit swarming capabilities onto fleets of preexisting UAVs that otherwise do not cooperate now.  Since the model only requires one-way communication it can be integrated into UAV software as an augmentation to the preexisting command input process instead of as an invasive modification to the core UAV software. This self-imposed constraint reduces the cost and design complexity of adding swarming behavior to legacy UAV systems compared to other swarming algorithms that require N-way multi-step conversations.

As shown by simulation the territorial swarming model is functional but there is always room for improvement.  In this scenario all aircraft had an equal communication range.  An interesting excursion would be to model varying communication ranges amongst individuals in the swarm.  Similarly, the addition of non-RF based communication payloads whose sole purpose is to relay communications from one part of the swarm to another would add an interesting dynamic.

There is room for enhancement in the rules for swarm cooperation.  Currently the same platform cannot Monitor and Attack at the same time even when both tasks encompass the same target.  This would certainly effect the effectiveness of the extremely small communication range cases.  Another interesting concept is the idea of an ``interested bidder'' for tasks.  In this case a platform is currently engaged in performing task A but hears about new task B.  The UAV is very well suited for task B but it should not abandon its current task A until someone can take it over.  As is some other UAV will pick up task B, but it may do the job poorly compared to the UAV stuck on task A.  With this new concept the busy UAV could announce that it is very interested in task B and that anyone who can closely match its score on task A can takeover the work for task A.  Other UAVs hearing about the interested bidder might change their behaviors to help relieve the UAV on task A or temporarily delay their execution of task B to see if the UAV on task A gets relieved.

The weighted average used for merging belief data in section~\ref{sec:uavBelief} is brought about by a limitation of the representation of world uncertainty.  Future work could find new ways of measuring uncertainty or weighing in the expertise of a sensor in the belief model so that a more automated merging algorithm can be used.  As is, the operator must provide a configuration value to weight the merging of data which adds to the burden of using this model.

\textbf{TODO: REVISE HERE UP}
 
%The application of the model and analysis of the results provides a procedure for exploring the trade space between communication ranges and mission effectiveness.  Given basic performance data about the aircraft and payloads we can simulate and analyze mission effectiveness.  For the fleet of aircraft modeled in this experiment we found that a communication range that could only reach across $10\%$ of the mission area provided near identical performance to a perfect global communication network.  Even smaller ranges could still provide useful mission effectiveness.  This provides a welcome option for aircraft designers to reduce their hardware requirements and for mission planners to optimize platform survivability and sensor evasion.

%As is the model can provide useful insights but there is always room for improvement.  In this scenario all aircraft had an equal communication range.  An interesting excursion would be to model varying communication ranges amongst individuals in the swarm.  Similarly, the addition of non-RF based communication payloads whose sole purpose is to relay communications from one part of the swarm to another would add an interesting dynamic.

%There is room for enhancement in the logic of rules for swarm cooperation.  Currently the same platform cannot Monitor and Attack at the same time even when both tasks encompass the same target.  This would certainly effect the effectiveness of the extremely small communication range cases.  Another interesting dynamic is the concept of an ``interested bidder'' for tasks.  In this case a platform is currently engaged in performing task A but hears about new task B.  The UAV is very well suited for task B but it should not abandon its current task A until someone can take it over.  As is some other UAV will pick up task B, but it may do the job poorly compared to the UAV stuck on task A.  With this new concept the busy UAV could announce that it is very interested in task B and that anyone who can closely match its score on task A can takeover the work for task A.



%\section{Suggestions for Future Work}
%Targets that sense they are under surveillance or attack, and evade in response.

%False targets/non-combatants.

%UAVs with comm-relay payloads?

%Intelligently choose a search algorithm instead of random.


%Enable attack and monitor from same platform.

%Sensor type expertise should weigh into cell and target belief merging


%"Interested" agent bidding - those stuck doing another task at the moment but might want to switch or queue up if their bid is extremely good


%Decentralized pursuit/evasion as in sun\_andrew\_k\_200906\_masc\_thesis.pdf?
