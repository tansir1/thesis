
\subsection{Task allocation}
\label{sec:uncoordTaskingMyWork}

%\chapter{Task Allocation}
%\section{Task allocation using only one way communication}
%\section{Territorial task allocation}

The allocation scheme used in this work is more closely aligned to the fully decentralized end of the spectrum.  It models the behaviors of territorial animals that shout, posture, or otherwise mark their domains so that other members of their species do not encroach.  The benefit of this behavior is that only one way communication is required.  There's no need for sequential data handling protocols.  If other agents exist within communications range they will share their belief models but they will not otherwise directly engage each other.  

Periodically agents analyze their belief model to find the best action they can do.  There are only two strongly enforced rules to control who performs a particular task.  

\begin{enumerate}
	\item The agent with the best Monitor or Attack score in the Target Belief data structure gets to perform the task.
	\item Once an Attack task's state is set to \textit{Performing} no other agent may claim the task regardless of rule 1.
\end{enumerate}

Rule 1 states that a ``survival of the fittest'' situation occurs.  If multiple UAVs compete for the same task then the strongest agent wins the task.  For purposes of this dissertation the scores are computed based on the UAV's ability to perform the task and their current distance from the task.  While it is theoretically possible for ties in scoring to occur in practice it is inconsequential.  The ability to perform a task has many variables and distances to the targets are always changing.  Each UAV's score will change slightly at the next periodic update.  Even so, the scoring metric described here is unimportant.  All that matters is that all UAVs use the same scoring process so that they can assess each other's relative strength and suitability.  Users of the algorithm are able to change to any other scoring metric that they choose.

When a target's Task State changes to \textit{NO TASK}, \textit{Open}, or \textit{Enroute} any UAV may attempt to claim it.  Claiming occurs when an agent is close enough to begin performing the action and they set the target's Task State to \textit{Performing}.  At all other times swarm agents are uninterested or interested in the task.  Uninterested agents do nothing in regards to the task because they either know they are too weak (relative to the current strongest agent who is Enroute) or are busy with another task already.  Interested agents will have the locally highest score for the task.  

Interested agents will set their Target Belief data with their computed score, their UAV ID, and change the Task State to Enroute.  The agent will then begin moving towards the task.  It's possible that multiple swarm agents can interested and Enroute to the same task simultaneously.  This occurs when communication constraints prevent the UAVs from knowing about each other or if multiple agents decide to pursue a task in between periodic broadcasts.  When multiple interested UAVs are Enroute they will eventually get close enough that their beliefs propagate through the network and merge with each other's.  When an interested agent learns about another UAV that is Enroute to a task the weaker one will back down and find another task to perform in order to comply with Rule 1.  The dominant UAV effectively guards their territory like an animal until a stronger one appears by posturing themselves with the best score.  The flight time while Enroute to a task is a potential competition period that emulates the advantages of a centralized controller or auctioneer in that an agent coordinates who gets to perform the task.  In this case the coordinating agent changes swiftly based on who the dominant UAV is at the time.

An agent may choose to stop performing a task by resetting the target's Task State to \textit{Open}.  When an agent claims an attack task rule 2 comes into effect and no other UAV may take the task.  UAVs performing the Monitor task may be overridden if another UAV comes within range with a significantly better score.  The definition of significantly better is definable by the operator as a percentage.   This helps to prevent churn in the monitoring process but allows a much better sensor to be utilized if it presents itself.  Either way, someone is always watching the target.
%\textbf{TODO: write out algorithm for task claiming}

These rules only work in an environment where no agent lies.  An agent can have out of date information and faulty sensors but they cannot actively lie.  All information is timestamped using a synchronized clock so stale data is a recoverable error and accounted for in all comparison and merging processes.  Faulty sensors can be overcome by other members of the swarm repeatedly contradicting the bad data.

Target Task States start in the \textit{NO TASK} state when the target is first discovered.  The discovering agent will create a target, initialize the Task States,  add the target to their belief model, and broadcast the model update out locally.  Other agents may hear about the new target, update their belief model, and rebroadcast it out again when they next periodically transmit their belief model.  In this fashion the task will propagate through the network.  However, there is a latency in the logic processing between agents near the task and those farther away.  It takes time for the task to hop from the interior to the exterior of the swarm.  During this timespan agents closer to the task may already begin to posture to claim it while those farther away are just first hearing about it.  Therefore it is recommend to have a weighting factor in the swarm's scoring function to specify if the operator prefers the closest UAV to perform the task or if they prefer the best equipped UAV to perform the task.  This weighting factor allows UAVs far away from the initial task announcement a chance to claim the task.

These two simple rules create a ``survival of the fittest'' mechanism that has many benefits in a poor communication environment.  Since all coordination is implied instead of being explicitly negotiated very little communication is required and no bidirectional communication occurs at all.  UAVs share their belief model of the world which is just a static data dump with well defined merging procedures.  Other UAVs hear this whole data packet or they hear nothing at all.  There is no need to maintain and manage temporary communication channels to facilitate negotiations or transactions like in market based algorithms.  This mechanism is most closely related to pheromone models with each UAV having their own unique pheromone signature.  This means the control laws for global navigation of the aircraft consolidate down to a series of strength comparisons with no higher level logic.  The formal algorithm for this task allocation scheme is written in Algorithm~\ref{alg:taskAlloc} of Appendix~\ref{sec:algorithms}.

%\textbf{TODO: 1 way useful for integrating into existing hardware with minimal software changes, make swarm out of hardware not meant to be a swarm, easier integration with legacy systems. Weighted average on merging because getting bayesian probability data for all other platforms and updating fleet is a difficult task. Higher quality platforms will weight their data higher than others in the fleet, could vary per target type?}
%\textbf{TODO: This is the whole point of thesis}
