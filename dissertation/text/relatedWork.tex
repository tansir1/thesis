\chapter{Related Work}
\todo{Add discussion about the other related works.}

A paper in IEEE Transactions written by \cite{jin} was the primary inspiration for the research described in this thesis.  The paper describes a ``\ldots heterogeneous team of cooperating UAVs drawn from several distinct classes and engaged in a search and action mission over a spatially extended battlefield with targets of several types.'' They created a simulation of multiple cooperating UAVs searching for multiple targets.  Each type of UAV was suited for a particular category of tasks and each type of target had a corresponding task or set of tasks that had to be completed upon it.  Some targets had known locations and others had to be found via cooperative search.  Tasks to be completed in the environment were cooperatively assigned in real time as the mission progressed.  Their algorithm had to balance the need to explore and the need to complete tasks using the appropriate type of UAV.  The model incorporated realism by stochastically determining when targets were found and if tasks were completed successfully.

While the model by \cite{jin} is fairly comprehensive there is room for improvement.  Jin’s model describes a military application of UAVs but assumes that the sensors are static and forward facing only.  Many military UAVs have gimbal mounted sensors.  The ability to aim the sensor without changing the trajectory of the host aircraft should affect the performance of task completion by the UAVs.  Their paper refers to the Pursuit-Evasion problem but there’s no evasion aspect to the model.  All the targets were statically located and never moved.  Lastly, the model is centralized but remarked as being convertible to a decentralized model.  The work in this thesis created a decentralized model.

A group from the Massachusetts Institute of Technology performed a similar experiment to Jin’s.   They too combined path planning and task allocation in \cite{bellingham}.  Their work is different in that target locations are known a priori and they wanted to minimize overall mission time.  Jin’s model is a reactive model in that the UAVs actively explore the spatial area whereas Wheeler’s model is proactive in that waypoints are generated between targets and UAVs before trajectories are planned.  A navigation graph is generated of all possible waypoints, target locations, and UAV locations and a search is run to find the shortest path between each UAV and all targets.  Then a task allocation algorithm is run to determine the most optimal UAV tasking based on time to completion and utilization of resources.  Neither model is superior to the other but each has a definitive purpose.  Jin’s model is optimized for unstructured ad-hoc search type missions whereas Wheeler’s model is optimized for structured quick strike type missions.

Another group combined path planning and task assignment in \cite{beard}.  Their model is different from \cite{jin} and \cite{wheeler} because they had only a single set of targets and tasks all known a priori.  Their goal was to create cooperative assignments such that multiple vehicles complete the same task(s) simultaneously.  The idea was to model a surprise attack situation in which multiple UAVs strike multiple targets at once giving preference to assigning multiple UAVs to strike high value targets.  Their model generated teams of UAVs, assigned them to targets, and generated flight paths such that all UAVs arrived at their targets at the same time while avoiding localized short-range anti-air defenses.

The previous groups have all done software simulations for their models and make assumptions about the real world and the ability of UAVs to fly in close proximity without issue.  The Insitu ScanEagle (a UAV) manufacturer and a group of universities teamed up to create a flock of UAVs that could pursue and track a moving ground target in \cite{wheeler}.  The Insitu ScanEagle is one of the smallest military UAVs with an inertially stabilized gimballed camera.  Ideally combining multiple sensors readings from different perspectives of the same target should create a more accurate measurement than a single sensor can provide.  Their goal was to maintain a precise geo-location track of the target and generate a history of its locations.  Their simulation and flight tests prove that a small swarm (or flock) of UAVs can cooperatively fly closely together without colliding while exchanging information.  The exchanged information included flight trajectories and information about the target. 

The biggest drawback of the work done by the previous groups is that all of their models are centralized.  There is a single routing point for all of the information about the world, tasks, targets, and UAVs.  The research described in this thesis  builds upon these previous groups but creates a decentralized system.  A decentralized system does not have a single point of failure and allows the swarm to continue to function in communication limited or denied environments which greatly increases the real-world applicability of the system compared to a centralized version.
