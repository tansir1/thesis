%\chapter{Target Model}
\section{Target Model}
\label{sec:target_model}
The simulation contains static targets and moving targets.  Static targets have a fixed position and orientation like a building.  Moving targets, as their name implies, have a dynamic position and orientation.  Targets have two other attributes: \textit{Best Angle} and \textit{Max Speed}.

The \textit{Max Speed} of a target is a measure of how fast it moves through the simulated worlds.  The \textit{Max Speed} of a static target is zero and a moving target has a non-zero positive value.  Moving targets travel through the road network from haven to haven with a fixed velocity similar to an automobile in a city.

The \textit{Best Angle} of a target is the best azimuthal angle for sensing and attacking the target as measured in the target's frame of reference.  The \textit{Best Angle} is used in computing the performance or chance of success of a weapon or sensor accomplishing its task against the target.  If the relative angle between the weapon or sensor and the target deviates from the \textit{Best Angle} then the weapon or sensor will suffer a performance degradation.  This is explained in more detail in section~\ref{sec:payload_probs}.  

The \textit{Best Angle} mechanic is the simulation's mechanism for modeling sensor performance characteristics of detecting, recognizing, and identifying a target based on the viewing angle and aspect ratio of the target.  Every type of target seen through an electroptical sensor needs a certain number of pixels before a computer or person can detect, recognize, and identify the observed entity.  The metrics and criteria for determining the number of required pixels was first proposed in \parencite{johnson_criteria} and commonly known today as Johnson Criteria.  More modern updates have furthered these concepts in \parencite{sensor_perf} and \parencite{evolution_johnson_criteria}.  The \textit{Best Angle} mechanic is a simplification for simulation purposes of a complex phenomena.

%A \textit{Best Angle} of $0^{\circ}$ is directly ahead on at the target.  Targets are assumed to be symmetrical so a \textit{Best Angle} of $90^{\circ}$ means the target can be best observed or attacked from its direct left or right.    

Moving targets cannot sense or actively evade UAVs.  They travel from haven to haven where they cannot be monitored or attacked by a UAV.  Once they arrive at a haven they will stay for a random but limited amount of time before moving to a new haven.  This behavior is designed to mimic a target hiding, performing deliveries, or having meetings at various locations.  The order of haven visitation is random.  Moving targets follow paths along the road network from haven to haven using the classic A* algorithm described in~\parencite{astar}.

