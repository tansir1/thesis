\chapter{Scenario Model and Description}

\colorbox{yellow}{This whole chapter is copy/paste from my proposal.  Clean it up.}


As in (Jin, et al. 2006) three styles of task assignment algorithms will be compared.  The base line tasking algorithm will allocate tasks in real-time as they appear.   A predictive algorithm will attempt to predict a chain of tasks before they occur and move members of the swarm into position ready to perform a task before it is generated.  This could occur when a target is first seen but not yet confirmed.   As a sensing aircraft confirms the target identity and orientation an attack aircraft could move into position and be prepared to strike after the confirmation task is complete.  The predictive algorithm causes a trade-off between search (exploration) and response (exploitation).  A third algorithm is a hybrid of the baseline and predictive algorithms that will try to merge the best of both algorithms.  It will default to exploring as in the baseline until certain criteria about the world are known and then it will switch to a predictive behavior until the criteria are no longer satisfied.

Payloads are one of the most costly elements on aircraft and the more functionality a payload has the more it typically costs.  As noted earlier, Jin’s model assumed only forward facing sensors.  One case study for this project will be to compare the performance of the three algorithms using static forward facing sensors against the performance of using gimballed sensors.  Intuitively gimballed sensors should provide more flexibility in the trajectories of aircraft allowing them to reach new tasks faster.  This case study will confirm whether this is true or not and if it is true will determine if gimballed sensors have enough of a benefit to warrant the additional cost, power requirements, and weight added to an aircraft.  

Additionally the performance comparison between gimballed and fixed sensors will be done over a wide range of swarm sizes.  It’s likely that there is a critical point where the size of a swarm is large enough that fixed sensors are equivalent in performance to fewer gimballed sensors.  In essence this will look for the point where the performance of many cheaper UAVs is comparable to fewer more expensive UAVs.

The second case study will explore the effects upon performance due to varying intra-swarm communication ranges.  The ranges will vary from no intra-swarm communication to a global communication range.  The control parameters for this study are the average density of the swarm per unit area and the range of communication.  Plotting the average density of the swarm, the communication range of the members, and the average total mission times will produce a three dimensional manifold.  The equation describing the shape of this manifold can be used to predict the number of UAVs needed to minimize total mission time given the communication range limits of the UAVs.

The data produced by both case studies can also be used for another purpose.  Each measures how the size of the swarm affects mission efficiency.  Based on this data we should be able to draw a conclusion on whether it’s more efficient to utilize a UAV’s size, weight, and power budget on better communication systems or better sensors.

The innovation in this project comes from creating a system that can readily be applied to real world situations.  The goal is not to create a better Pursuit-Evasion or Weapon Target Assignment solution.  The goal is to create a system that can merge the two problems, regardless of how those subdomains are solved, within a limited or disrupted communication environment.  This will advance the scientific domain of operations research for multi-agent coordinated mission and resource management.

Coordinating the efforts of multiple aircraft in a small airspace is difficult and dangerous.  Add in limited or disrupted communications and the amount of risk involved dramatically increases.  This mission management work requires many people to operate together and all the supporting equipment that goes with it. The algorithm generated by this research can automate this coordination process.  Automation will allow much faster and much more accurate modeling of the region compared to a group of people.  The algorithm can serve as air traffic controllers, pilots, and mission controllers simultaneously.  This means aircraft can fly closer together allowing more aircraft to be utilized simultaneously in a small area increasing the probability of mission success in a timely fashion.

The results of this work should make it possible for a single person to direct a swarm of UAVs.   This has applicability to military, disaster response, search and rescue, border protection, aerial firefighting, and precision agriculture scenarios.

The two case studies focus on how the size of the swarm affects mission performance.  The results will help future swarm creators determine how best to design a swarm.  Should fewer more advanced aircraft be used or should many simple aircraft be used?  The data from the case studies can help answer these questions.   Furthermore the equations derived from the communication manifold case can be used to estimate how many UAVs are needed in a swarm to complete the objectives within a timely manner given the characteristics of the swarm’s aircraft.  It can also be used as a predictive model for determining the probability of a success of a mission after losing members of a swarm.  The swarm operators should know the available flight time or operational time left in the swarm before the aircraft run out of fuel or power.  Comparing this time against the predicted time to complete a mission could provide a rough estimate on the probability of mission success.